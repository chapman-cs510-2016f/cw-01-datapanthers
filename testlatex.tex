\documentclass[aps,pra,notitlepage,amsmath,amssymb,letterpaper,12pt]{revtex4-1}
\usepackage{amsthm}
\usepackage{graphicx}
 
%  Helpful commands to set up problem environments easily
\newenvironment{problem}[2][Problem]{\begin{trivlist}
\item[\hskip \labelsep {\bfseries #1}\hskip \labelsep {\bfseries #2.}]}{\end{trivlist}}
\newenvironment{solution}{\begin{proof}[Solution]}{\end{proof}}
 
% --------------------------------------------------------------
%                   Document Begins Here
% --------------------------------------------------------------
 
\begin{document}
 
\title{Classwork 1}
\author{Simrath Ratra, Andrew Nguyen, Afnan Alqahtani}
\affiliation{CS510 Computing for Scientists}
\date{\today}

\maketitle

% x.yz is the problem number
\begin{problem}{1}
 What does derivative f'(x) of a function f(x) mean exactly?
\end{problem}
 
\begin{solution} %Explaining what F(x)' means

A derivative shows the change in the curve at any point of the curve; 
an instantaneous rate of change of a function. 

By definition, the derivative of f(x) with respect to x is the function f'(x) and is defined: 

\[ f'(x) \lim_{h \to 0} = \frac{f(x+h)-f(x)}{h} \]



For example, take the equation: 
\[ f(x) = x^2\]


\begin{figure}[h!] % h forces the figure to be placed here, in the text
  \includegraphics[width=0.4\textwidth]{derivative_ex.jpg}  % if pdflatex is used, jpg, pdf, and png are permitted
  \caption{Solving the derivative using our formula.}
  \label{fig:figlabel}
\end{figure}

What we have found is that the derivative \[ f(x) = x^2\] at a point x, is the slope of the line tangent to the curve of \[ y = x^2\]. So for example, if x = 1, then the slope of our tangent line will be 2. 

\begin{figure}[h!] % h forces the figure to be placed here, in the text
  \includegraphics[width=.6\textwidth]{tangent.jpg} 
  \caption{Tangent Line and its slope.}
  \label{fig:figlabel}
\end{figure}

This text should be below the figure unless \LaTeX decides that a different layout works better.
\end{solution}
 
% Repeat as needed
 
 
\end{document}

%%% INSTRUCTOR COMMENTS %%%
%
% Very good.  Remember to remove parts of the document that you don't need, when using a template like this. 
% Also, when putting math inline, you can use $math here$. This would be useful for $f(x)$ or $f'(x)$, 
% for example, so that they are formatted the same way as the rest of the math.
%
%%%%%%%%%%%%%%%%%%%%%%%%%%%
